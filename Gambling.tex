\documentclass{article}

\begin{document}
\begin{titlepage}
\begin{center}
\line(1,0){300}\\
[0.3in]
\huge{\bfseries MAKERERE UNIVERSITY
COLLEGE OF COMPUTING AND INFORMATICS TECHNOLOGY\\
BIT 2207 Research Methodology\\
Lecturer: Mr. Ernest Mwebaze }\\
[1.5cm]
\textsc{\LARGE NYANZI ALLAN\\
RegNo: 16/U/997\\
StudentNo: 216001149}\\
\end{center}
\end{titlepage}
\tableofcontents
\thispagestyle{empty}
\cleardoublepage
\setcounter{page}{1}

\section{Intoduction}\label{sec:intro}
PROBLEM GAMBLING


Problem gambling (or ludomania, but usually referred to as "gambling addiction" or "compulsive gambling") is an urge to gamble continuously despite harmful negative consequences or a desire to stop. Problem gambling is often defined by whether harm is experienced by the gambler or others, rather than by the gambler's behaviour. Severe problem gambling may be diagnosed as clinical pathological gambling if the gambler meets certain criteria. Pathological gambling is a common disorder that is associated with both social and family costs.
\section{signs of a gambling problem}
Lying\\
Chasing losses\\
Borrowing money\\
Always betting more\\
Being obsessed with gambling\\
Being unable to stop gambling

\subsection{Lying}

People who have gambling problems generally try to hide it from the people around them. They start lying to their spouses, families, co-workers, and friends.

\subsection{Chasing losses}

Some gamblers say they are just trying to win back the money they have lost. They will claim that once they win big, they will stop. Or that they lost because they changed strategies or were not lucky. But when they chase their losses, they end up piling up even more losses, and often debts.

\subsection{Borrowing money}

What do pathological gamblers do when gambling puts them into a financial hole? They borrow—from their family, friends, co-workers, or even strangers, without always admitting the real reason they need the money. They may also have other people pay their gambling debts. They may max out their credit cards or take out a second mortgage

\subsubsection{Always betting more}

Like someone who has drugs or alcohol problems, problem gamblers have to up their dose of gambling to enjoy it. In other words, they have to bet more and more money to get the kind of rush they want. Unfortunately, the more they bet, the more they lose

\subsection{Being obsessed with gambling}

When this happens, gamblers cannot stop thinking about the last time they gambled and the next time they will. Any reason is reason enough to go gambling, and they will try any strategy to get the money they need.

\subsection{Being unable to stop gambling}

Many gamblers know they should not gamble so much, and want to quit. They try repeatedly, but cannot fight the urge to play.


\section{How to overcome a Gambling Addiction}


Avoid isolation. Call a trusted family member, meet a friend for coffee, or go to a Gamblers Anonymous meeting.



Distract yourself with another activity, such as going to the gym, watching a movie, or practising a relaxation exercise for gambling cravings.



Postpone gambling. Tell yourself that you'll wait 5 minutes, fifteen minutes, or an hour. As you wait, the urge to gamble may pass or become weak enough to resist.


Visualize what will happen if you give in to the urge to gamble. Think about how you'll feel after all your money is gone and you've disappointed yourself and your family again.



Seek help for underlying mood disorders. Depression,  stress, substance abuse, or anxiety can both trigger gambling problems and be made worse by compulsive. 






\end{document}